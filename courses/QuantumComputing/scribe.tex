\documentclass{article}
\usepackage[margin=1in]{geometry} % for margins
\usepackage[dvipsnames,svgnames]{xcolor} % for colors
\usepackage[
  colorlinks=true,
  linkcolor=SteelBlue,
  citecolor=SteelBlue,
  urlcolor=SteelBlue
]{hyperref} % for links and citations

\setlength{\parindent}{0pt}

%%%%%%%%%% FONT PACKAGES %%%%%%%%%%
\usepackage{libertine} % for libertine font family
\usepackage{amsmath,amssymb,amsthm} % for mathematical symbols
\usepackage{braket} % for braket notation 
\usepackage{mdframed} % for box 
\usepackage{algorithm} % for pseudocode 
\usepackage{algpseudocode}



%%%%%%%%%% THEOREM ENVIRONMENTS (GLOBAL NUMBERING) %%%%%%%%%%

% Master counter
\newtheorem{theorem}{Theorem}

% Share the same counter
\newtheorem{lemma}[theorem]{Lemma}
\newtheorem{proposition}[theorem]{Proposition}
\newtheorem{corollary}[theorem]{Corollary}

\theoremstyle{definition}
\newtheorem{definition}[theorem]{Definition}
\newtheorem{example}[theorem]{Example}

\theoremstyle{remark}
\newtheorem{remark}[theorem]{Remark}



%%%%%%%%%% MATH SYMBOLS %%%%%%%%%%
\newcommand{\C}{\mathbf{C}}
\newcommand{\R}{\mathbf{R}}
\newcommand{\F}{\mathbf{F}}
\renewcommand{\phi}{\varphi}
\newcommand{\eps}{\varepsilon}

%%%%%%%%%% MACROS %%%%%%%%%%
\newcommand{\coursetitle}{COL7160 : Quantum Computing}
\newcommand{\instructor}{Rajendra Kumar}
\newcommand{\scribe}{Poojan Shah} % write your name here 
\newcommand{\lecturenumber}{0} % write lecture number here 
\newcommand{\lecturetitle}{Scribe Template} % write lecture title here


\begin{document}

 %%%%%%%%%% TITLE %%%%%%%%%%
\begin{mdframed}
\begin{center}
    {\Large {\coursetitle}} \\[0.5em]
    {\large {Lecture \lecturenumber : \lecturetitle}}
\end{center}

\vspace{0.8em}

\noindent
\begin{minipage}{0.48\textwidth}
\textbf{Instructor:} \instructor
\end{minipage}
\hfill
\begin{minipage}{0.48\textwidth}
\raggedleft
\textbf{Scribe:} \scribe
\end{minipage}
\end{mdframed}
\vspace{1em}

%%%%%%%%%% BEGIN WRITING HERE %%%%%%%%%%
\section{Introduction}

This is the scribe template for COL7160 : Quantum Computing. If you are not familiar with \LaTeX, you can go through \href{https://www.overleaf.com/learn/latex/Learn_LaTeX_in_30_minutes}{this tutorial} about \LaTeX and Overleaf. For the template, you have been provided with two files : you will write the main content in $\mathtt{scribe.tex}$ and add references (if necessary) to $\mathtt{refs.bib}$. For citing references, you can use the $\mathtt{\\cite}$ command. For example, the main references for this course will be \cite{dewolf2023quantumcomputinglecturenotes} and \cite{nielsenchuang2010}. 

\section{Mathematical Notation}

To ensure that the math in all the scribe notes is consistent, we have provided a list of commands that would be useful throughout the course. Real numbers are denoted by $\R$. Complex numbers are denoted by $\C$. For any $z = x+iy\in \C$, its modulus is $|z| = \sqrt{x^2+y^2}$ and its conjugate is $z^* = x-iy$. The vector space of all $n$ tuples of complex numbers $(z_1,\dots,z_n)$ is denoted by $\C^n$. A vector is denoted by a \emph{ket} $\ket{\psi} \in \C^n$. The vector \emph{dual} to $\ket{\psi}$ is represented by the \emph{bra} $\bra{\psi}$. The inner product between the vectors $\ket{\phi}$ and $\ket{\psi}$ is denoted by $\braket{\phi|\psi}$. $\ket{\phi}\otimes\ket{\psi}$ is the tensor product of $\ket{\phi}$ and $\ket{\psi}$. For a matrix $A$, $A^*$ is the complex conjugate of $A$, $A^\top$ is the transpose of $A$ and $A^\dagger$ is the Hermitian conjugate or adjoint of $A$. 

\section{Definitions and Theorems}

Throughout the scribe notes, we will use standard theorem-like environments
to clearly distinguish between definitions, formal statements, and explanatory
remarks. Below we illustrate how these environments are used.

\begin{definition}
A \emph{vector space} over a field $\F$ is a set $V$ together with two operations,
called vector addition and scalar multiplication, satisfying the usual axioms
(linearity, associativity, distributivity, and the existence of additive
identities and inverses).
\end{definition}

\begin{example}
The set $\C^n$ of $n$-tuples of complex numbers forms a vector space over $\C$
with component-wise addition and scalar multiplication.
\end{example}

\begin{lemma}
Let $V$ be a vector space and let $v \in V$. If $\alpha v = 0$ for some scalar
$\alpha \in \F$, then either $\alpha = 0$ or $v = 0$.
\end{lemma}

\begin{proof}
Suppose $\alpha \neq 0$. Then $\alpha^{-1}$ exists and
\[
v = \alpha^{-1}(\alpha v) = 0.
\]
\end{proof}

\begin{theorem}
Any finite-dimensional vector space admits a basis.
\end{theorem}

\begin{remark}
Remarks are used for intuition, informal explanations, or connections to
other parts of the course. They are not formal statements and typically do not
require proofs.
\end{remark}

\medskip

Unless stated otherwise, all definitions and theorems are numbered
consecutively throughout the document, independent of section numbering.


\section{Algorithms}

Algorithms should be written in pseudocode.

\begin{algorithm}[H]
\caption{Gram-Schmidt Orthonormalization}
\begin{algorithmic}[1]
\State \textbf{Input:} A linearly independent set
$\{\ket{w_1}, \ket{w_2}, \dots, \ket{w_d}\}$
in a finite-dimensional inner-product space $V$.
\State \textbf{Output:} An orthonormal basis
$\{\ket{v_1}, \ket{v_2}, \dots, \ket{v_d}\}$ spanning $V$.
\State $\ket{v_1} \gets \ket{w_1} / \|\ket{w_1}\|$

\For{$k = 1$ to $d-1$}
    \State $ \ket{u_{k+1}} \gets \ket{w_{k+1}}-\sum_{i=1}^{k}
    \braket{v_i \mid w_{k+1}}\,\ket{v_i}$
    \State
    $ \ket{v_{k+1}} \gets \frac{\ket{u_{k+1}}}{\|\ket{u_{k+1}}\|} $
\EndFor

\State \Return $\{\ket{v_1}, \ket{v_2}, \dots, \ket{v_d}\}$
\end{algorithmic}
\end{algorithm}


\bibliographystyle{alpha}
\bibliography{refs}

\end{document}